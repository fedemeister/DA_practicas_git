Mi implementación del algoritmo de ordenación por fusión (decreciente) para objetos Celda es la siguiente:
\begin{lstlisting}
// Algoritmo basado en el pseudocodigo del libro "Introduction to Algorithms 3era edicion pagina 31"
void merge(std::vector<Celda> &V, int p, int q, int r) {
  int n1 = q - p + 1;
  int n2 = r - q;
  std::vector<Celda> L(n1), R(n2);
  for (int i = 0; i < n1; ++i) {
    L[i].valor_ = V[p + i].valor_;
  }
  for (int j = 0; j < n2; ++j) {
    R[j].valor_ = V[q + j + 1].valor_;
  }
  int i = 0;
  int j = 0;
  int z = p;
  while (i < n1 && j < n2) {
    if (L[i].valor_ >= R[j].valor_) { // ahora ordena de forma decreciente.
      V[z].valor_ = L[i].valor_;
      i++;
    } else {
      V[z].valor_ = R[j].valor_;
      j++;
    }
    z++;
  }
  while (i < n1) {
    V[z].valor_ = L[i].valor_;
    i++;
    z++;
  }
  while (j < n2) {
    V[z].valor_ = R[j].valor_;
    j++;
    z++;
  }
}
\end{lstlisting}

\begin{lstlisting}
// Algoritmo basado en el pseudocodigo del libro "Introduction to Algorithms 3era edicion pagina 31"
void mergeSort(std::vector<Celda> &V, int p, int r) {
  if (p < r) {
    int q = (p + r) / 2;
    mergeSort(V, p, q);
    mergeSort(V, q + 1, r);
    merge(V, p, q, r);
  }
}
\end{lstlisting}