Nuestros algoritmos de colocación de defensas tendrán que colocar las defensas en el mapa indicado. Nosotros hemos representado el terreno como una matriz cuadrada y hemos almacenados las celdas disponibles en un vector.
\begin{itemize}
	\item Sin preordenar. Nuestro algoritmo tendrá que ir recorriendo todas las celdas del terreno de batalla tantas veces como defensas haya disponible. Por lo tanto tendrá que recorrer ese vector en el mejor y peor caso. O(n2)
	\item Usando algoritmos de ordenación por fusión y ordenación rápida. Ambos ordenarán ese vector una vez rellenado con las celdas candidatas con O(n log n), pero con al ventaja que para seleccionar una celda con mayor valor siempre es O(1) al estar ordenado.
	\item Usando montículo / cola de prioridad. En esta ocasión el vector se va ordenando cada vez que se inserta y se extrae un elemento con un O(n log n). Teóricamente este método es óptimo cuando puede entrar una celda en cualquier momento y se ordenaría "sola" en lugar de tener que llamar otra vez a la función para que lo ordenase.
\end{itemize}
Como veremos en la gráfica del ejercicio siguiente, tener que buscar un número mucho menor de defensas en algunos mapas que el número de celdas disponibles hace que sea conveniente usar un algoritmo de colocación de defensas sin ordenación y buscar las pocas veces que haga falta colocar una defensa entre las celdas disponibles en lugar de ordenar un vector enorme para no sacarle mucho partido. También hay que tener en cuenta que el algoritmo con ordenación o montícuo serán mejores opciones dependiendo de si en ese mapa va a ser dificil colocar las defensas en las celdas con mayor valoración cuando sea dificil que sean celdas factibles (ahí sí se sacaría partido al tener las celdas ordenadas e ir seleccionando en orden constante).