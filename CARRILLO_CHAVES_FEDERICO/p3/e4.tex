\begin{lstlisting}
bool funcionComparacion(Celda const &celda1, Celda const &celda2) {
  return celda1.valor_ > celda2.valor_;
}

\end{lstlisting}

\begin{lstlisting}
void pruebasCajaNegra() {
  // Parametros que recibe celda es valor_ y Vector3_ respectivamente.
  Celda c1(10, 0), c2(20, 0), c3(30, 0), c4(40, 0);
  Celda c5(50, 0), c6(60, 0), c7(70, 0);
  std::vector<Celda> v{c7, c6, c5, c4, c3, c2, c1};
  std::vector<Celda> copia_de_v(v.size());
  copia_de_v = v;
  std::vector<Celda> aux_mergeSort(v.size());
  std::vector<Celda> aux_quickSort(v.size());
  unsigned int ok = 0;
  unsigned int mal = 0;
  bool b;
  do {
    aux_mergeSort = v;
    aux_quickSort = v;
    mergeSort(aux_mergeSort, 0, v.size() - 1);
    quickSort(aux_quickSort, 0, v.size() - 1);
    b = true;
    for (int i = 0; i < v.size(); ++i) {
      if (aux_mergeSort[i].valor_ != copia_de_v[i].valor_ || aux_quickSort[i].valor_ != copia_de_v[i].valor_)
        b = false;
    }
    ++(b ? ok : mal);
  } while (next_permutation(v.begin(), v.end(), funcionComparacion));
  std::cout << "Permutaciones OK: " << ok << ", Permutaciones Error: " << mal << std::endl;
}
\end{lstlisting}
\begin{quote}
    Permutaciones OK: 5040, Permutaciones Error: 0    
\end{quote}
