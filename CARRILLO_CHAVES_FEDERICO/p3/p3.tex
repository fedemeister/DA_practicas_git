\documentclass[]{article}

\usepackage[left=2.00cm, right=2.00cm, top=2.00cm, bottom=2.00cm]{geometry}
\usepackage[spanish,es-noshorthands]{babel}
\usepackage[utf8]{inputenc} % para tildes y ñ
\usepackage{graphicx} % para las figuras
\usepackage{xcolor}
\usepackage{listings} % para el código fuente en c++

\lstdefinestyle{customc}{
  belowcaptionskip=1\baselineskip,
  breaklines=true,
  frame=single,
  xleftmargin=\parindent,
  language=C++,
  showstringspaces=false,
  basicstyle=\footnotesize\ttfamily,
  keywordstyle=\bfseries\color{green!40!black},
  commentstyle=\itshape\color{gray!40!gray},
  identifierstyle=\color{black},
  stringstyle=\color{orange},
}
\lstset{style=customc}


%opening
\title{Práctica 3. Divide y vencerás}
\author{Nombre Apellido1 Apellido2 \\ % mantenga las dos barras al final de la línea y este comentario
correo@servidor.com \\ % mantenga las dos barras al final de la línea y este comentario
Teléfono: xxxxxxxx \\ % mantenga las dos barras al final de la linea y este comentario
NIF: xxxxxxxxm \\ % mantenga las dos barras al final de la línea y este comentario
}


\begin{document}

\maketitle

%\begin{abstract}
%\end{abstract}

% Ejemplo de ecuación a trozos
%
%$f(i,j)=\left\{ 
%  \begin{array}{lcr}
%      i + j & si & i < j \\ % caso 1
%      i + 7 & si & i = 1 \\ % caso 2
%      2 & si & i \geq j     % caso 3
%  \end{array}
%\right.$

\begin{enumerate}
\item Describa las estructuras de datos utilizados en cada caso para la representación del terreno de batalla. 

Para el centro de extracción de minerales, en adelante C.E.M, las celdas del terreno del mapa estarán mejor valoradas cuánto más cerca se encuentren a la posición central del mapa.
% Elimine los símbolos de tanto por ciento para descomentar las siguientes instrucciones e incluir una imagen en su respuesta. La mejor ubicación de la imagen será determinada por el compilador de Latex. No tiene por qué situarse a continuación en el fichero en formato pdf resultante.
%\begin{figure}
%\centering
%\includegraphics[width=0.7\linewidth]{./defenseValueCellsHead} % no es necesario especificar la extensión del archivo que contiene la imagen
%\caption{Estrategia devoradora para la mina}
%\label{fig:defenseValueCellsHead}
%\end{figure}

\item Implemente su propia versión del algoritmo de ordenación por fusión. Muestre a continuación el código fuente relevante. 

Nuestra tabla de subproblemas resueltos (en adelante TSR) se define de la siguiente manera: \\
\begin{equation*}
    TSR_{m,n} = 
    \begin{pmatrix}
    a_{1,1} & a_{1,2} & \cdots & a_{1,n} \\
    a_{2,1} & a_{2,2} & \cdots & a_{2,n} \\
    \vdots  & \vdots  & \ddots & \vdots  \\
    a_{m,1} & a_{m,2} & \cdots & a_{m,n} 
    \end{pmatrix}
\forall m,n \in \mathbb{Z^+}
\end{equation*}
siendo m = número de defensas y n = ases. \newline

Dentro de ella guardaremos una serie de valores. Esos valores verán determinados por el valor o el peso de la defensa, que están guardados un std::vector de objetoMochila.\\
Dicho objetoMochila es una clase donde guardaremos el valor, el coste y el identificador de cada defensa a partir de la $defensa_{1}$ , ya que la $defensa_{0}$ ya está en la lista de seleccionados.


\item Implemente su propia versión del algoritmo de ordenación rápida. Muestre a continuación el código fuente relevante. 

Escriba aquí su respuesta al ejercicio 3.

\item Realice pruebas de caja negra para asegurar el correcto funcionamiento de los algoritmos de ordenación implementados en los ejercicios anteriores. Detalle a continuación el código relevante.

\begin{lstlisting}
bool funcionComparacion(Celda const &celda1, Celda const &celda2) {
  return celda1.valor_ > celda2.valor_;
}

\end{lstlisting}

\begin{lstlisting}
void pruebasCajaNegra() {
  // Parametros que recibe celda es valor_ y Vector3_ respectivamente.
  Celda c1(10, 0), c2(20, 0), c3(30, 0), c4(40, 0);
  Celda c5(50, 0), c6(60, 0), c7(70, 0);
  std::vector<Celda> v{c7, c6, c5, c4, c3, c2, c1};
  std::vector<Celda> copia_de_v(v.size());
  copia_de_v = v;
  std::vector<Celda> aux_mergeSort(v.size());
  std::vector<Celda> aux_quickSort(v.size());
  unsigned int ok = 0;
  unsigned int mal = 0;
  bool b;
  do {
    aux_mergeSort = v;
    aux_quickSort = v;
    mergeSort(aux_mergeSort, 0, v.size() - 1);
    quickSort(aux_quickSort, 0, v.size() - 1);
    b = true;
    for (int i = 0; i < v.size(); ++i) {
      if (aux_mergeSort[i].valor_ != copia_de_v[i].valor_ || aux_quickSort[i].valor_ != copia_de_v[i].valor_)
        b = false;
    }
    ++(b ? ok : mal);
  } while (next_permutation(v.begin(), v.end(), funcionComparacion));
  std::cout << "Permutaciones OK: " << ok << ", Permutaciones Error: " << mal << std::endl;
}
\end{lstlisting}
\begin{quote}
    Permutaciones OK: 5040, Permutaciones Error: 0    
\end{quote}


\item Analice de forma teórica la complejidad de las diferentes versiones del algoritmo de colocación de defensas en función de la estructura de representación del terreno de batalla elegida. Comente a continuación los resultados. Suponga un terreno de batalla cuadrado en todos los casos. 

Escriba aquí su respuesta al ejercicio 5.

\item Incluya a continuación una gráfica con los resultados obtenidos. Utilice un esquema indirecto de medida (considere un error absoluto de valor_ 0.01 y un error relativo de valor_ 0.001). Es recomendable que diseñe y utilice su propio código para la medición de tiempos en lugar de usar la opción \emph{-time-placeDefenses3} del simulador. Considere en su análisis los planetas con códigos 1500, 2500, 3500,..., 10500, al menos. Puede incluir en su análisis otros planetas que considere oportunos para justificar los resultados. Muestre a continuación el código relevante utilizado para la toma de tiempos y la realización de la gráfica.

\begin{lstlisting}
struct ComparaValor2 {
  bool operator()(Celda const &celda1, Celda const &celda2) {
    // Cuanto menor valor, mayor prioridad.
    // Con esto conseguimos que las celdas con menor distancia al CEM tengan mas prioridad.  
    return celda1.valor_ < celda2.valor_;
  }
};

\end{lstlisting}

\begin{lstlisting}
int cellValue(int row, int col, bool **freeCells, int nCellsWidth, int nCellsHeight, float mapWidth, float mapHeight, const std::list<Object *> &obstacles, std::list<Defense *> defenses, bool esCEM) {
  .
  .
  .
  if (esCEM) {
      .
      .
      .
  } else {
    int i_out;
    int j_out;
    std::list<Defense *>::const_iterator ci_CEM = defenses.begin();
    Vector3 aux_vector3;
    positionToCell((*ci_CEM)->position, i_out, j_out, cellWidth, cellHeight);
    aux_vector3 = cellCenterToPosition(row - i_out, col - j_out, cellWidth, cellHeight);
    return std::max(mapWidth, mapHeight) - aux_vector3.length();
  }
}
\end{lstlisting}

\begin{lstlisting}
void DEF_LIB_EXPORTED
placeDefenses(bool **freeCells, int nCellsWidth, int nCellsHeight, float mapWidth, float mapHeight, std::list<Object *> obstacles, std::list<Defense *> defenses) {
    .
    .
    .
// rest of our defenses
std::priority_queue<Celda, std::vector<Celda>, ComparaValor2> Q2; // nueva cola de prioridad
 for (int i = 0; i < nCellsHeight; i++) {
  for (int j = 0; j < nCellsWidth; j++) {
    aux_vector3 = cellCenterToPosition(i, j, cellWidth, cellHeight);
    Celda aux_celda = Celda(
        cellValue(i, j, freeCells, nCellsWidth, nCellsHeight, mapWidth,  mapHeight, obstacles, defenses, false), 
        aux_vector3
        );
    Q2.push(aux_celda);
    }
 }
    
  while (defensaCandidata != defenses.end()) {
    Celda celda = Q2.top();
    Q2.pop();
    (*defensaCandidata)->position = celda.vector3_;
    if (factibilidad(defensesPlaced, mapWidth, mapHeight, (*defensaCandidata), obstacles)) {
      defensesPlaced.push_back((*defensaCandidata));
      ++defensaCandidata;
    }
  }
}
\end{lstlisting}

\end{enumerate}

Todo el material incluido en esta memoria y en los ficheros asociados es de mi autoría o ha sido facilitado por los profesores de la asignatura. Haciendo entrega de este documento confirmo que he leído la normativa de la asignatura, incluido el punto que respecta al uso de material no original.

\end{document}
