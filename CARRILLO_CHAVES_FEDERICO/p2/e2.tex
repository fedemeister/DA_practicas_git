Nuestra tabla de subproblemas resueltos (en adelante TSR) se define de la siguiente manera: \\
\begin{equation*}
    TSR_{m,n} = 
    \begin{pmatrix}
    a_{1,1} & a_{1,2} & \cdots & a_{1,n} \\
    a_{2,1} & a_{2,2} & \cdots & a_{2,n} \\
    \vdots  & \vdots  & \ddots & \vdots  \\
    a_{m,1} & a_{m,2} & \cdots & a_{m,n} 
    \end{pmatrix}
\forall m,n \in \mathbb{Z^+}
\end{equation*}
siendo m = número de defensas y n = ases. \newline

Dentro de ella guardaremos una serie de valores. Esos valores verán determinados por el valor o el peso de la defensa, que están guardados un std::vector de objetoMochila.\\
Dicho objetoMochila es una clase donde guardaremos el valor, el coste y el identificador de cada defensa a partir de la $defensa_{1}$ , ya que la $defensa_{0}$ ya está en la lista de seleccionados.