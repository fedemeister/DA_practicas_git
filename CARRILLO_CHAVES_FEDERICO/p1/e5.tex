Sabemos que el C.E.M es la primera posición de la lista de defensas colocadas, por lo que obtenemos esa celda mirando la primera posición de esa lista.
A través de la función positionToCell(...) conseguimos sacar la fila i y la columna j en la que está nuestro C.E.M.
Y sacamos la distancia de la celda candidata a nuestro C.E.M.
Posteriormente calculamos la distancia de esa celda candidata con nuestro C.E.M. Cuanto menor sea la distancia, mejor.
Con esto haremos que las defensas rodeen a nuestro C.E.M protegiéndolo.