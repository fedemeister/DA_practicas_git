\documentclass[]{article}

\usepackage[left=2.00cm, right=2.00cm, top=2.00cm, bottom=2.00cm]{geometry}
\usepackage[spanish,es-noshorthands]{babel}
\usepackage[utf8]{inputenc} % para tildes y ñ

%opening
\title{Práctica 4. Exploración de grafos}
\author{Nombre Apellido1 Apellido2 \\ % mantenga las dos barras al final de la línea y este comentario
correo@servidor.com \\ % mantenga las dos barras al final de la línea y este comentario
Teléfono: xxxxxxxx \\ % mantenga las dos barras al final de la linea y este comentario
NIF: xxxxxxxxm \\ % mantenga las dos barras al final de la línea y este comentario
}


\begin{document}

\maketitle

%\begin{abstract}
%\end{abstract}

% Ejemplo de ecuación a trozos
%
%$f(i,j)=\left\{ 
%  \begin{array}{lcr}
%      i + j & si & i < j \\ % caso 1
%      i + 7 & si & i = 1 \\ % caso 2
%      2 & si & i \geq j     % caso 3
%  \end{array}
%\right.$

\begin{enumerate}
\item Comente el funcionamiento del algoritmo y describa las estructuras necesarias para llevar a cabo su implementación.

Para el centro de extracción de minerales, en adelante C.E.M, las celdas del terreno del mapa estarán mejor valoradas cuánto más cerca se encuentren a la posición central del mapa.
% Elimine los símbolos de tanto por ciento para descomentar las siguientes instrucciones e incluir una imagen en su respuesta. La mejor ubicación de la imagen será determinada por el compilador de Latex. No tiene por qué situarse a continuación en el fichero en formato pdf resultante.
%\begin{figure}
%\centering
%\includegraphics[width=0.7\linewidth]{./defenseValueCellsHead} % no es necesario especificar la extensión del archivo que contiene la imagen
%\caption{Estrategia devoradora para la mina}
%\label{fig:defenseValueCellsHead}
%\end{figure}

\item Incluya a continuación el código fuente relevante del algoritmo.

Nuestra tabla de subproblemas resueltos (en adelante TSR) se define de la siguiente manera: \\
\begin{equation*}
    TSR_{m,n} = 
    \begin{pmatrix}
    a_{1,1} & a_{1,2} & \cdots & a_{1,n} \\
    a_{2,1} & a_{2,2} & \cdots & a_{2,n} \\
    \vdots  & \vdots  & \ddots & \vdots  \\
    a_{m,1} & a_{m,2} & \cdots & a_{m,n} 
    \end{pmatrix}
\forall m,n \in \mathbb{Z^+}
\end{equation*}
siendo m = número de defensas y n = ases. \newline

Dentro de ella guardaremos una serie de valores. Esos valores verán determinados por el valor o el peso de la defensa, que están guardados un std::vector de objetoMochila.\\
Dicho objetoMochila es una clase donde guardaremos el valor, el coste y el identificador de cada defensa a partir de la $defensa_{1}$ , ya que la $defensa_{0}$ ya está en la lista de seleccionados.


\end{enumerate}

Todo el material incluido en esta memoria y en los ficheros asociados es de mi autoría o ha sido facilitado por los profesores de la asignatura. Haciendo entrega de esta práctica confirmo que he leído la normativa de la asignatura, incluido el punto que respecta al uso de material no original.

\end{document}
